\section{Conclusion}
\label{chap:conclusions}
At the beginning of this work, I asked myself several questions so that I could focus my work in a specific way. 
The questions I asked myself are:
\begin{itemize}
\item Does it make sense in 2021 to move to digital or is the government as we know it now okay as it is? 
\item Has the mandatory distancing helped authorities to digitalize?
\item Is it clear now to the authorities and the population the value of electronic government?
\item Might this actually be a path to a possible future?
\end{itemize}

I tried to answer these questions as best I could, trying to understand how the government is structured, and seeing where we started with e-government, what steps were taken, where we are now, and what we are focusing on; this topic was quite complicated to research because it's a gigantic topic, and by searching on official channels the information spreads out tremendously like in anything government related. And I feel like I've scratched the tip of this huge iceberg.
My results show that Switzerland started relatively late compared to other EU countries, this delay has penalized the country that in a hurry is trying to get back on track in recent years.
In order to catch up with other nations, innovative reforms with young political and technical groups were needed to modernize the idea of old government with old rules.
Technology has made great strides over the past decade, this improvement has benefited the various strategies, three at the present time, actually making it easier to implement and work on those technologies.
Unfortunately, there have been a number of problems, the most serious of which is electronic voting, which at the time of writing is still not working. The electronic vote has been slowed down or even stopped, and this is very serious in a country where voting and direct democracy are the basis of everything.
In this last year then there has been one more difficulty, the covid-19 hit the whole world certainly not helping the situation of electronic government, actually slowing down everything.
I think that man is able to adapt very quickly, always or almost always finding a solution to the problem that bothers him, or at least to limit the biggest problems.
The pandemic caused general lockdowns and freezes to everyday life, but it was also an all-around test when it comes to the IT world. At the beginning of the pandemic, there were several problems, but it was also possible to understand the strengths and limitations that e-government has. Most of the people in government are part of the generation before mine, with covid many were not prepared for this digital boom and remote working, certainly adapting has been more difficult for them than for younger people who are used to technology.
The crisis will certainly encourage the development of these digital tools, and I hope that there is a change of mentality to technology and what it can do, we need a push in electronic voting by bringing this mode first in government and then to all the people, with the ability to be identified and vote directly remotely.
In conclusion, Switzerland was not ready for the covid, in fact at the time when the digital government could give its best it could not do much better, because it did not come prepared with several weaknesses in digital government, the covid opened the Pandora's box on eGov highlighting with insistence all the weaknesses and in fact showing the momentary inability for Switzerland to have a digital government.
Just look at countries like Estonia, where the government is 99\% digital, it is also true that it is 1/8th of Switzerland in terms of population and this certainly makes it easier to manage. More than comparison we should take it as an example.
My hope is that we can learn from the mistakes made and quickly improve the situation at least on the fundamental points because this present, in my opinion, could actually be our future, it can and must be improved.