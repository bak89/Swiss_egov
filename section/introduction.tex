\section{Introduction}
\epigraph{\centering \textit{“The beauty of e-governance is that a few keystrokes can bring smiles on a million faces.”}}{Narendra Modi}

The following work attempts to accomplish tasks that have been assigned to me:
\begin{quotation}
\textit{Produce a comprehensive state-of-progress (past, present, future) report on Swiss e-government. COVID has shown that the digital transformation of public services is a pressing need.}
\end{quotation}
So I'm going to try to get to the bottom of the eGovernment topic. First, the reasons why I chose this task.
As a foreigner I want to understand how the country I live in works. It is for me a fundamental part of integration, unfortunately the country where I come from is famous for its low-level governments. Based on this precondition my interest is to see if technology can actually help a country to improve itself, and if something has actually changed since e-gov was implemented.\\
When we talk about government, we often do it in a negative way.
We always feel the need for something to improve, what if this something is digitization?\\
Today's world is becoming more and more virtual, you can do just all sorts of things from the internet, so why not also make the government new, that is the brain that makes a country move?\\
In this report we will see how the Swiss government has moved more and more to digitization and see the benefits it has provided.\\
We are in 2020 and this year the most commented theme is definitely the pandemic.\\
The COVID-19 has expanded all over the world arriving also in Switzerland, with serious consequences including a lockdown.\\
In the lock-down time, we could all experience distance and isolation; at this point the digital world came to help us.\\
Indeed, if before the pandemic topics such as e-school, e-government or e-voting were seen as something distant in time, now it is an increasingly possible and necessary reality.\\
\newpage
\subsection{Country Profile}

\begin{table}[h!]
%\renewcommand\arraystretch{1.2}
%\begin{table}[]
\centering
%\resizebox{\textwidth}{!}{%
\begin{tabular}{|l|l|}
\hline
\multicolumn{2}{|c|}{\textbf{Country Profile}}                    \\ \hline
Population (1 000):   & 8 606 inhabitants (2019)         \\ \hline
\acrshort{gdp} at market prices: & 703.08 million Euros (2019)      \\ \hline
\acrshort{gdp} growth rate:      & -7.3\% *(Jul 2020)               \\ \hline
Inflation rate:       & -06\% *(Oct 2020)                \\ \hline
Unemployment rate:    & 3.2\% *(Oct 2020)                \\ \hline
Area:                 & 41,285 km2                       \\ \hline
Capital city:         & Bern                             \\ \hline
Official language:    & German, French, Italian, Romansh \\ \hline
Currency:             & \acrfull{chf}                             \\ \hline
Head of State: & \href{https://www.admin.ch/gov/en/start/federal-council.html}{Federal Council}\cite{federalcouncil} \\ \hline
Head of Government: & \href{https://www.admin.ch/gov/en/start/departments/department-of-environment-transport-energy-communications-detec.html}{President Simonetta Sommaruga}\cite{simonetta} \\ \hline
\end{tabular}
%}
\caption{Country Profile}
\label{tab: profile}
\end{table}

\begin{table}[h!]
\centering
%\resizebox{\textwidth}{!}{%
\begin{tabular}{|c|l|l|}
\hline
\multicolumn{3}{|c|}{\textbf{Information Society Indicators}}                                                                                                                                                                                            \\ \hline
\multirow{2}{*}{\begin{tabular}[c]{@{}c@{}}Percentage of  \\ households with\end{tabular}}                      & Internet access in Switzerland:                                                                               & 96\% (2019)   \\ \cline{2-3} 
                                                                                                                & \begin{tabular}[c]{@{}l@{}}broadband connection \\ in Switzerland:\end{tabular}                               & 99.9\% (2016) \\ \hline
\multirow{4}{*}{\begin{tabular}[c]{@{}c@{}}Percentage of \\ individuals using \\ the internet for\end{tabular}} & \begin{tabular}[c]{@{}l@{}}interacting with public \\ authorities in Switzerland:\end{tabular}                & 75\% (2019)   \\ \cline{2-3} 
                                                                                                                & \begin{tabular}[c]{@{}l@{}}downloading official forms from \\ public authorities in Switzerland:\end{tabular} & 58\% (2019)   \\ \cline{2-3} 
                                                                                                                & \begin{tabular}[c]{@{}l@{}}sending filled forms from \\ public authorities in Switzerland:\end{tabular}       & 45\% (2019)   \\ \cline{2-3} 
                                                                                                                & \begin{tabular}[c]{@{}l@{}}obtaining information from \\ public authorities in Switzerland:\end{tabular}      & 68\% (2019)   \\ \hline
\end{tabular}%
%}
\caption{Information Society Indicators}
\label{tab: indicators}
\end{table}

source:Eurostat (last update: 20 May 2020), \href{https://tradingeconomics.com/switzerland/gdp-growth#:~:text=GDP\%20Growth\%20Rate\%20in\%20Switzerland,the\%20second\%20quarter\%20of\%202020.}{Tading Economics*}